\chapter{Conclusion}

In this thesis we have proposed efficient algorithms for graph node
similarity and decision tree learning. Our goal was to create
algorithms that scale out regardless of input size and to do so
we set three design goals: reduce the amount of computation
necessary to produce a result, reduce the amount of communication
during distributed learning, and bound the memory use of the algorithms,
making it possible to train them on unbounded datasets.

In this chapter we conclude this dissertation, by first providing a
summary and critical view of the results, and putting our research
in context with the state of the art.
We close the chapter with potential future work directions and open
problems.

\section{Summary of Results}
\todo{Haven't really placed the work in the context of SOTA here...}
We first proposed an approximate way to calculate similarities between
nodes in a graph, and discover concepts in the derived similarity
graph. Our method dramatically reduces the computational cost
by localizing computation in the nodes' neighborhoods, and taking advantage
of the sparse graph structure to further optimize the computation with controllable
error. We demonstrated the generality of the approach in several domains and presented
qualitative as well as quantitative evidence of its accuracy. In addition, we demonstrate
the scalability of the algorithm by training on one of the biggest text datasets
available in minutes.

We then presented a real-world use case of music streaming session length analysis and prediction,
which demonstrated the unique characteristics of music streaming listeners, and developed
an appropriate predictive model. This work acts as a motivation for our follow-up work
on decision tree learning.

For our first contributions to decision tree learning,
we presented two algorithms to extract uncertainty estimates
for the predictions of ensembles of online regression trees.
We use approximate data structures and online learners to
adapt two batch algorithms to the online domain, bounding
their computational and memory use and making it possible
to train the algorithms on unbounded streaming data. We demonstrate the
favorable performance of the algorithm against the state of
the art in terms of accuracy, and an order of magnitude improvement
in the runtime.

Our next contribution again deals with online decision tree learning, where we develop an algorithm
to learn boosted online decision trees in parallel, achieving significant
learning speedups while maintaining the correctness guarantees of the
underlying online boosting algorithms. Our approach scales almost linearly,
for both weak and strong scalability and we clearly demonstrate the gains
in terms of accuracy over single trees and speed compared to single-threaded
approaches.


Finally we presented a new algorithm for distributed gradient boosted tree
learning that enables a new dimension of scalability, allowing for scale-out
across both the data point and feature dimensions. We make use of the Quickscorer
algorithm to achieve efficient block-distributed prediction, and make use
of the parameter server's flexibility to exploit data sparsity and achieve
order of magnitude improvements in the communication cost of the training process.

\section{Discussion}

We would to discuss here the extent to which this work answers the research
question posited for our thesis which we reproduce here:

\begin{displayquote}
	\researchQuestion
\end{displayquote}

\noindent
This dissertation has tackled this question by:
\begin{itemize}
	\item Using approximate solutions and bounded error subsampling of edges
	to produce a vertex similarity algorithm with a scalable
	distributed implementation.

	\item Using approximate data structures and online learning methods
	to enable the estimation of online random forest uncertainty. The approximations
	made trade off convergence guarantees for efficient computation.

	\item Using distributed computation to tackle high-dimensional problems
	for boosted decision trees, maintaining the correctness of the algorithms
	while providing significant speedups. We make clear the limitations
	of the approaches for low dimensional and dense data.
\end{itemize}


\section{Future Directions}

We identify several aspects of our work that can be expanded upon and improved,
targeting specific limitations of the model. For our work on graph node similarity
our next targets are to create hierarchical structures of concepts, thereby creating
composable representations in an unsupervised manner. The main challenge for this
task is again the computational cost, as we would need to maintain information
about the role of each node at each hierarchy level as we propagate the relationships.
Another interesting problem is developing
a generalized algorithm for evaluating graph node similarity from a stream of edges,
that form the graph in real time. Finally the calculation of similarities between
nodes in the graph that are more than two hops away is currently, by design, not included
in the model. However, there could be a way to calculate these by following paths in
the resulting similarity graph.

For our work in the estimation of uncertainty in online decision trees, our foremost
priority would be the theoretical justification of the validity of the produced intervals,
i.e. to prove that, like the batch QRF and conformal prediction, the produced intervals
are consistent estimators given the requested significance level. The ability to
deal with concept drift at the meta-algorithm level would also be welcome, instead
of relying on the underlying weak learner to provide this capability.

For our work parallel boosted online trees, one weakness are datasets with
a small number of features, which limits the potential speedup of the
algorithm.
An interesting direction to explore is to
create efficient data-parallel and block-parallel algorithms that might sacrifice
the correctness guarantees of our feature-parallel algorithm for more available
parallelism.

Finally for our work on block-distributed gradient boosted trees one priority is to
improve the algorithm's performance for datasets with low sparsity. The overhead
introduced by the sparse representations and distributed communication systems we use could be mitigated by,
for example, choosing the representation to use (sparse or dense) based on the
characteristics of the data, which can be determined in the initial pass over the
data during the creation of the feature value histograms. In addition we would
like to integrate the approach in an end-to-end learner like XGBoost to be able
to compare its performance with other state-of-the-art systems like LightGBM.
