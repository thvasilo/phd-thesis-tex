\chapter{Conclusion}

In this thesis we have proposed efficient algorithms for graph node
similarity and decision tree learning. Our goal was to create
algorithms that scale out regardless of input size and to do so
we have used three design goals: reduce the amount of computation
necessary to produce a result, reduce the amount of communication
during distributed learning, and bound the memory use of the algorithms,
making it possible to train them on unbounded datasets.

In this chapter we conclude this dissertation, by first providing a
summary and critical view of the results, and putting our research
in context with the state of the art.
We close the chapter with potential future work directions and open
problems.

\section{Summary of Results}

We first proposed an approximate way to to calculate similarities between
nodes in a graph. Our method dramatically reduces the computational cost
by localizing computation in node neighborhoods, and taking advantage
of the sparse graph structure to further optimize the computation with controllable
error. We demonstrated the generality of the approach in several domains and presented
qualitative as well as quantitative evidence of its accuracy. In addition we demonstrate
the scalability of the algorithm by training on one of the biggest text datasets
available in minutes.

We then presented a real-world use case of music streaming session length analysis and prediction,
which demonstrated the unique characteristics of music streaming listeners, and developed
an appropriate predictive model. This work acts as a motivation for our follow-up work
on decision tree learning.

For our first contribution to decision tree learning, we develop an algorithm
to learn boosted online decision trees in parallel, achieving significant
learning speedups while maintaining the correctness guarantees of the
underlying online boosting algorithms. Our approach scale almost linearly,
for both weak and strong scalability and we clearly demonstrate the gains
in terms of accuracy over single trees and speed compared to single-threaded
approaches.

Our next contribution again deals with online decision tree learning,
and we presented two algorithms to extract uncertainty estimates
for the predictions of ensemble of online regression trees.
We use approximate data structures and online learners to
adapt two batch algorithms to the online domain, bounding
their computational and memory use and making it possible
to train the algorithm on unbounded data. We demonstrate the
favorable performance of the algorithm against the state of
the art in terms of accuracy, and an order of magnitude improvement
in the runtime.

Finally we presented a new algorithm for distributed gradient boosted tree
learning that enable a new dimension of scalability, allowing for scale-out
across both the data point and feature dimensions. We make use the Quickscorer
algorithm to achieve efficient block-distributed prediction, and make use
of the parameter server's flexibility to exploit data sparsity and achieve
order of magnitude improvements in the communication cost of the training process.

\section{Future Directions}

We identify several aspects of work that can be expanded upon and improved,
targeting specific limitations of the model. For our work on graph node similarity
our next targets are to create hierarchical structures of concepts, thereby creating
composable representations in an unsupervised manner. We would also like to create
a generalized algorithm for evaluating graph node similarity from a stream of edges,
that form the graph in real time. Finally the calculation of similarities between
nodes in the graph that are more than two hops away is currently, by design, not included
in the mode. However, there could be a way to calculate these by following paths in
the resulting similarity graph. An efficient algorithm to do that would generalize
the problem the algorithm is able to solve.

For our work parallel boosted online trees, one limitation is for datasets with
a limited amount of features, which would limit the potential speedup of the
algorithm.
An interesting direction to explore is to
create efficient data-parallel and block-parallel algorithms that might sacrifice
the correctness guarantees of our feature-parallel algorithm for more available
parallelism.

For our work in the estimation of uncertainty in online decision trees, our foremost
priority would be the theoretical justification of the validity of the produced intervals,
i.e. to prove that, like the batch QRF and conformal prediction, the produced intervals
are consistent estimators given the requested significance level. The ability to
deal with concept drift at the meta-algorithm level would also be welcome, instead
of relying on the underlying weak learner to provide this capability.

Finally for our work on block-distributed gradient boosted trees one priority is to
improve the algorithm's performance for datasets with low sparsity. The overhead
introduced by the sparse representations and distributed communication systems we use could be mitigated by,
for example, choosing the representation to use (sparse or dense) based on the
characteristics of the data, which can be determined in the initial pass over the
data during the creation of the feature value histograms. In addition we would
like to integrate the approach in an end-to-end learner like XGBoost to be able
to compare its performance with other state-of-the-art systems like LightGBM.
