\chapter{Introduction}

Machine learning is now is being introduced into more and more areas in
our lives. What once was mostly used in the realm of computer science for
tasks like digit recognition and automated call centers is now being used
in medicine, engineering, agriculture, climate science, aeronautics,
and finance\todo{Cite all!}. This explosive growth of machine learning has originated
to a large part from the dramatic decrease in the cost of storing and processing
data, and much of the accuracy gains for machine learning models these
days is made possible by this abundance of data.

However, more data means more processing cost which leads to the need for
machine learning algorithms that can be trained \emph{efficiently} on massive
datasets. Efficiency can be interpreted in different ways. One
aspect is the ability of an algorithm to take full advantage of modern
hardware architectures. As CPU clock speed increases have stalled, developers
and algorithm designers have turned to parallelism as a means of
extracting more performance out of existing hardware.
For gathering and processing the massive datasets available today,
shared-nothing clusters of commodity machine have risen to
be the dominant choice in the industry due to both locality
in terms of taking the computation to an already distributed
dataset, as well as cost-effectiveness. These distributed
architectures introduce additional challenges on top of
learning in parallel on a single machine.

Another efficiency factor that has been becoming more important
is learning under a tight computational budget. Whether that
is learning at the edge on networks of IoT devices or federated
learning with the intent of protecting user privacy, there has
been an increasing demand for algorithms that can be trained
with bounded compute and memory resources.
These use-cases also highlight another challenge in learning
in a constantly evolving environment, where traditional batch
models trained on historical data will under-perform as the
relationships between the features and dependent evolve.
%Online learning under a tight computational budget is the challenge
%of streaming learning, which has been a focus for this dissertation.

Many systems and algorithms have been developed in the past to
tackle these issues. The popularity of deep learning~\cite{deep-learning}
in particular has led to an deluge of research on efficient
algorithms and systems. These include systems like Tensorflow, MXNet,
and PyTorch and have expanded to hardware specialization with recent projects
like TVM and Glow. While these systems are optimized for the gradient
descent optimization that neural networks are commonly trained with
this thesis is focused instead on two other models. The first comes
from the data mining domain and deals with the determining the
similarity between nodes in a graph, based on structural information.
For this purpose algorithms like SimRank and its variants have been
proposed, but since they all try to solve the all-to-all similarity
problem directly, their scalability suffers for massive graphs.
In our approaches we instead approximate the problem by focusing
our computation on local neighborhoods, dramatically reducing the
computational cost to arrive at an approximate solution.

The second model we focus on in this thesis are decision trees,
both in the batch and online domain. In the batch domain we investigate
gradient boosted trees in particular, whose popularity has also led
to the development of multiple scalable systems like XGBoost~\cite{xgboost},
LightGBM~\cite{lightgbm}, and CatBoost~\cite{catboost}. What all these
systems have in common is that they only allow scale-out across
the data point dimension, meaning that in order to scale learning
for high-dimensional data we need to scale-up to bigger, more expensive
machines, a limitation which we address in our research. Apart from
batch decision trees, online decision trees have been proposed
to bring the accuracy and interpretability of decision trees to
the online domain~\cite{vfdt}. However, the concessions made to
bring the learning algorithm to the online domain has led to
decreased accuracy in practice, as well as the inability to apply
techniques from the batch domain to, for example, estimate the uncertainty
in the predictions. In our research we tackle the accuracy problem for
massive streaming data by developing efficient online boosted trees,
and propose online algorithms to estimate the uncertainty in the
predictions of tree ensembles.

These challenges set the stage for this thesis: We propose scalable algorithms
for learning from massive data efficiently. We utilize different techniques
to deal with the data size including designing efficient parallel and distributed
algorithms, making use of approximations to dramatically reduce the computational
cost, and approximate data structures to bound the memory and computational cost
of keeping models up to date. Throughout the development of this thesis the guiding
principle has been to design algorithms that can scale regardless of data size.
This is realized through algorithms that exhibit linear scale-out characteristics,
bounding the memory and computational costs through approximations, and designing
algorithms that are optimized for distributed settings through communication-efficient
learning.

\section{Research Questions \& Objectives}

The objective of this dissertation is to create scalable algorithms for machine
learning. Our approach has been to use two aspects to ensure scalability:
approximation from the algorithmic side, and parallelism from the system
side. By approximation we mean either reducing an intractable problem
to a more easily solved one that provides a useful answer, or by producing
online models which can be trained on datasets of arbitrary size using bounded
compute and memory. From the system side we develop algorithms that are parallel
and some that are specifically optimized for the distributed setting.
Our approach is to design solutions and provide implementations
in popular open-source frameworks like Apache Spark~\cite{spark},
MOA~\cite{samoa}, and XGBoost~\cite{xgboost}, ensuring the reproducibility
of our work, and further contributing to the open-source community.

In order to create algorithms that scale to massive datasets and are able to
do so efficiently in a distributed setting, we can identify three goals that can
lead to efficient solutions \cite{vasia-thesis}:

\begin{enumerate}
	\item Reduce the amount of computation.
	\item Reduce the amount of communication.
	\item Bound the space cost.
\end{enumerate}

Chapters \ref{ch:concepts}, \ref{ch:uncertain-trees}, and \ref{ch:boosted-trees}
are examples where approximation has been used to make a computationally intractable
approach scalable, reducing the total amount of computation.
In Chapter \ref{ch:concepts} for example, we provide an approximate solution
to the all-to-all graph node similarity problem by using the structure of the
graph to limit the computational cost. While the produced result cannot give
us a similarity score between any two arbitrary nodes in a graph, the end result captures
much of the relevant information and provides results that can be of great use
to practitioners.

Chapters \ref{ch:concepts} and \ref{ch:boosted-trees} describe learning algorithms that
are designed from the ground up with the distributed setting in mind. As network
communication is a sparse resource in cluster environments \cite{optimization-communication-complexity},
our algorithms aim to minimize communication cost while providing valid
solutions for the problems of graph node similarity calculation and gradient
boosted tree learning.

Chapter \ref{ch:uncertain-trees} and Section \ref{sec:boostvht-method}
are examples of online learning models
that were explicitly designed to use bounded compute and memory regardless of the
dataset size. For example, Chapter \ref{ch:uncertain-trees} provides online learning approaches
for methods that previously only existed in a batch setting, where the data were assumed to be
static and bounded, and ensures that the computational cost of the methods are
bounded in compute and memory. Approximations are made in this setting as well,
as we replace exact data structures with approximate ones, and trade-off consistency
guarantees for the ability to train the models online.


\subsection{Research Questions}

This thesis taken as a whole constitutes an effort of finding appropriate
uses of approximation in order to make exact methods scale to massive datasets.
In addition, in all our work we try to take full advantage of modern hardware,
whether that is providing parallel implementations of our algorithms,
or by designing algorithms from the
ground up to work and be efficient in a distributed setting.

These two aspects form the connecting thread between the papers
included in the thesis: Use of approximation to make previously expensive approaches tractable, and
proper utilization of modern hardware to ensure we make full use of current
computing capabilities and allow for scale-out with the problem size.

The overall research question that summarizes the objectives of this dissertation is the following:

\begin{displayquote}
	\emph{How can we use approximations and modern hardware to make otherwise computationally intractable
	approaches scale?}
\end{displayquote}

\noindent
Based on this research question, we develop our methods utilizing approximate computation and efficient parallel implementations. Our thesis statement is therefore the following:

\begin{displayquote}
	\emph{Approximations allow us to make otherwise computationally intractable approaches scalable.
	By carefully designing our trade-offs, we can extract useful results using a fraction
	of the resources exact approaches require. These approximations combined with utilizing modern hardware lead to efficient, scalable solutions.}
\end{displayquote}



\section{Methodology}

This research strategy for this dissertation is to perform quantitative, empirical research.
Our approach has been to perform a literature review to identify the challenges in the
state of the art, identify areas where approximation and parallelism can be applied to
optimize performance, and proceed with the design and implementation of the algorithms.

The algorithm design is guided by the research objectives, namely once we identify a problem
to work on, e.g. uncertainty estimation in online decision trees, we use the three objectives
listed as a guide for our optimizations. When exact solutions exist for a particular problem
or for a different domain, e.g. a solution exists for batch learning but not for the online
domain, we look to identify the approximations that will make the approach scale to massive
data efficiently. In particular we can identify different kinds of optimizations:

\begin{itemize}
	\item \emph{Replacing exact data structures with approximate ones.} This approach can be taken
	to improve both the computational and memory cost of existing approaches. Depending on the
	level of accuracy necessary for a particular estimator, many times approximate data structures
	can deliver similar accuracy to exact ones, at a fraction of the computational cost.
	One example in our research described in Chapter \ref{ch:uncertain-trees} is replacing the dense array of dependent values for the
	estimation of quantiles at tree leafs with quantile sketches, thereby bounding the
	memory cost of the algorithm.
	\item \emph{Using a proxy problem to tackle and intractable one.} This can mean solving a
	tractable proxy problem that can inform us about the true solution, as we have done
	in Chapter \ref{ch:concepts} where we provide an approximate efficient algorithm for the node similarity problem by localizing the computation to one-hop neighborhoods, dramatically reducing the
	amount of computation necessary to produce an approximate but informative answer.
	\item \emph{Adapting the algorithm to the data to improve accuracy and efficiency.} In many problems
	we have prior information about the data structure. For example many real-world graphs exhibit
	power-law degree distribution \cite{small-world}, many user interaction metrics are power-law
	distributed \cite{phonecalls, faloutsos1999internet, barabasi-small-world, click-stream-power-law} or exhibit extreme sparsity \cite{esl}. This properties can
	be exploited to reduce the computational and memory cost of learning algorithms as well as
	improve their accuracy. In Chapter \ref{ch:concepts} we rely on the power-law degree
	distribution of many real-world graphs to provide approximations with measurable error,
	making an otherwise intractable computation efficient. In Chapters \ref{ch:session-length}
	we adapt the objective function to better model the power-law distribution of the dependent.
	In Chapter \ref{ch:boosted-trees} we make use of data sparsity to dramatically reduce
	the communication cost of distributed algorithms for boosted trees.
\end{itemize}

\section{Contributions}

The main contributions of this dissertation, along the papers they appear in
are listed below:

\begin{itemize}
	\item \textbf{\conceptsicdm}: Knowing an object by the company it keeps: A domain-agnostic scheme
	for similarity discovery. \\
	\textbf{\conceptskais}: Domain-Agnostic Discovery of Similarities and Concepts at Scale.

	In these works we we deal with the following research questions:\\
	\emph{How to create a scalable way to calculate the similarity between nodes
	in a graph? How can we make effective use of distributed computing for this task?
	}

	All-to-all graph node similarity approaches, like the established SimRank algorithm~\cite{simrank},
	cannot scale to massive graphs due to the computational cost scaling with at least a quadratic
	factor in the number of nodes. Our approach is to approximate the all-to-all problem
	by limiting the similarity calculation to one-hop-neighborhoods, following the
	notion of context similarity as stated by Firth in \cite{firth} as ``You shall know
	a word by the company it keeps''. By limiting the amount of computation we introduce
	approximations with controllable error allowing for similarity calculations on massive
	data. Our implementation is optimized for the distributed setting, with its most
	expensive operation being a \emph{self-join} operation that has linear speed-up and
	constant scale-up, with limited communication.

	\item \textbf{\sessionlength}: Predicting Session Length in Media Streaming.

	In this work we deal with the following research question:\\
	\emph{Can we effectively predict the amount of time a user will spend using
	a music streaming application at the moment they first open it?}

	While session length distribution has been investigated for search
	queries and post-ad click behavior, the behavior of users in a media
	streaming service is likely to differ greatly.
	In this work we provide an analysis of the session length distribution
	of a major online music streaming service using tools from survival analysis, and develop an appropriate
	model to predict session length from a number of features including
	user-based and contextual session-based features.
	We demonstrate the differences in the way that sessions develop and end
	between users, and illustrate the importance of selecting an appropriate
	objective function for a non-negative, power-law distributed dependent value.
	This work act as a use-case for our follow up work, as it motivates the use of online learning,
	uncertainty estimation, and large-scale distributed learning with gradient boosted
	trees, topics we subsequently worked on in Papers \boostvhtNum, \uncertaintreesNum,
	and \blockgbtNum respectively.

	\item \textbf{\boostvht}: BoostVHT: Boosting Distributed Streaming Decision Trees.

	In this work we deal with the following research question:
	\emph{How can we make use of modern hardware to parallelize and distribute
	the computation of the otherwise serial online boosting algorithm?}

	While online decision trees allow us to maintain an up-to-date scalable
	model using bounded memory and computation, the approximations they make
	can lead to decreased accuracy.
	Boosting is one of the most successful ensemble techniques to increase the
	accuracy of weak learners. However, existing online boosting approaches
	are strictly sequential, making parallelization challenging, while existing batch parallel boosting algorithms
	require the data points to be processed simultaneously, breaking the assumptions
	of the online algorithms. In this paper we bridge
	this disconnect between online and parallel boosting, by introducing
	online parallel and distributed boosting, by parallelizing learning
	across the features, thereby maintaining the guarantees of online boosting
	algorithms while providing significant speedups.

	\item \textbf{\uncertaintrees}: Quantifying Uncertainty in Online Regression Forests.

	In this work we deal with the following research question:
	\emph{How can we estimate the uncertainty in the predictions of online tree
	ensemble methods?}

	Uncertainty estimation is of paramount importance when applying learning methods
	to domains where mistakes can be costly, like finance and autonomous vehicles.
	In addition, in these domains we have a constantly changing environment,
	and learning algorithms need to be able to constantly adapt. While ensemble
	of decision trees have been shown to be accurate estimators, they online
	counterparts lack any way to estimate the uncertainty in their predictions.
	In this paper we develop two general algorithms for uncertainty estimation
	from online random forest ensembles, adapting batch methods to the online
	domain through approximate computation.


	\item \textbf{\blockgbt}: Block-distributed Gradient Boosted Trees.


	In this work we deal with the following research questions:
	\emph{How can we improve the scalability of distributed gradient boosted tree learning
	in high dimensions? Can we add another scale-out dimension while keeping the
	communication costs manageable?}

	Gradient boosted decision trees (GBT) are one of the most popular algorithms in
	use in industry and research alike. Their popularity stems from their ability
	to deal with massive datasets, and several systems for distributed learning
	of GBTs have been developed. However one common aspect of these systems is that
	they only enable scale-out across the data point dimension, meaning that in
	order to speed-up learning per-feature we need to scale-up using bigger and more
	expensive machines. In this work demonstrate how to enable scale-out across
	both the data point \emph{and} feature dimensions, thereby allowing for scale-out
	across both, and by taking advantage of the structure of sparse datasets,
	enable faster and more cost-efficient training.
\end{itemize}

\subsection{Software}

As part of this dissertation we have worked on the following software:

\begin{itemize}
	\item XGBoost~\cite{xgboost} is the most popular gradient boosted trees library,
	used across academia and industry. As part of our work we have used XGBoost as the
	learning tool for \sessionlength, and have worked directly to extend XGBoost to
	support block-distribution
	for \blockgbt. As both of these works were performed during internships in private
	companies (Pandora Media and Amazon AWS respectively), we have not been able to release
	the code as open-source at the time of this writing.
	As a result of this work we have become actively involved in the wider
	development efforts of XGBoost and hope to integrate our block-distributed
	work in the future.
	\item MOA \cite{moa-book} is one of the most popular online learning library and includes
	support for training, evaluation and data generation for online/streaming learning.
	We have used MOA as a baseline comparison for \boostvht and extended it with
	support for uncertainty estimation among other improvements in \uncertaintrees,
	which we plan to contribute back to the project.
	\item Apache SAMOA \cite{samoa} is a library and framework that enable the development
	of distributed online learning algorithms, that can then be run on top of multiple
	stream processing engines like Apache Flink \cite{flink} and Apache Storm \cite{storm}.
	Our work on \boostvht has been performed on top of SAMOA and has been contributed back
	to the project.
\end{itemize}

Other software developed during our PhD studies that is not included in this
thesis however, is the distributed machine learning library for
Apache Flink, FlinkML, which is the official ML library for Apache Flink
since the 0.9 release\footnote{\url{https://flink.apache.org/news/2015/06/24/announcing-apache-flink-0.9.0-release.html}}.

\section{Related Work}

\section{Thesis Organization}

The rest of this dissertation is organized as follows:
In Part \ref{part:bg} of the thesis we provide background information and
related work relevant to the topics covered in the thesis. Chapter
\ref{ch:bg-parallel-ml} introduces some basic building blocks
for parallel and distributed machine learning, Chapter \ref{ch:bg-online-learning}
presents online learning, Chapter \ref{ch:bg-graph-similarity} describes the problem
of graph node similarity and existing approaches, and we end this part
with Chapter \ref{ch:bg-decision-trees} that acts as a brief introduction
in decision tree learning in the context of this thesis.

Part \ref{part:results} presents the results of the work developed for this
dissertation. We have tried to make each chapter self-sufficient, providing
brief explanations of the approaches along with the most significant results.
Papers \conceptsicdmNum and \conceptskaisNum are presented in Chapter \ref{ch:concepts},
Paper \sessionlengthNum is presented in Chapter \ref{ch:session-length}, and the final
two chapter focus on decision tree learning, with Paper \uncertaintreesNum
in Chapter \ref{ch:uncertain-trees} and Papers \boostvhtNum and \blockgbtNum
being presented in Chapter \ref{ch:boosted-trees}.

We finish the thesis with Part \ref{part:conclusion} with a conclusion and discussion
chapter.
