\chapter{Introduction}

Machine learning is now being introduced into more and more areas in
our lives. What once was mostly used in the realm of computer science for
tasks such as digit recognition \cite{mnist} is now being used
in medicine~\cite{health-ml-1, health-ml-2}, energy~\cite{ml-building, ml-energy},
agriculture~\cite{ml-agriculture, ml-agriculture-2},
climate science~\cite{ml-climate-science}, structural engineering~\cite{ml-engineering},
and finance~\cite{ml-finance}. This explosive growth of machine learning has originated
to a large part from the dramatic decrease in the cost of storing and processing
data. The accuracy gains for machine learning models these
days are made possible by this abundance of data in combination with sophisticated
models.

However, ``more'' data means ``more'' computation which leads to the need for
machine learning algorithms that can be trained \emph{efficiently} on massive
datasets. Efficiency can be interpreted in different ways. One
aspect is the ability of an algorithm to take full advantage of modern
hardware architectures. As CPU clock speed increases have stalled, developers
and algorithm designers have turned to parallelism as a means of
extracting more performance out of existing hardware.
For gathering and processing the massive datasets available today,
clusters of commodity machines have risen to
be the dominant choice in the industry due to both locality
in terms of taking the computation to an already distributed
dataset, as well as cost-effectiveness. These distributed
architectures introduce additional challenges on top of
learning in parallel on a single machine.

Another efficiency factor that has been becoming more important
is learning under a tight computational budget. Whether that is
learning at the edge on internet of things devices, or
federated learning with the intent of protecting user privacy, there has
been an increasing demand for algorithms that can be trained
with bounded computation and memory resources.
These use-cases also highlight another challenge which is learning
in a constantly evolving environment. Traditional batch
models trained on historical data can perform sub-optimally as the
relationships between the features and dependent evolve.
%Online learning under a tight computational budget is the challenge
%of streaming learning, which has been a focus for this dissertation.

Many systems and algorithms have been developed in the past to
tackle these issues. The popularity of deep learning~\cite{deep-learning}
in particular has led to a deluge of research on efficient
algorithms and systems. These include systems like Tensorflow~\cite{tensorflow},
MXNet~\cite{mxnet},
and PyTorch~\cite{pytorch} and have expanded to support heterogeneous hardware with recent projects
like TVM~\cite{tvm} and Glow~\cite{glow}. While these systems are optimized for the gradient
descent optimization that neural networks are commonly trained with,
this thesis is focused instead on two other models, \emph{graphs and decision
trees}.

The first problem we deal with originates
from the data mining domain and tackles the problem of determining the
similarity between nodes (vertices) in a graph, based on structural information,
and uncovering semantic concepts from related nodes.
For this purpose algorithms like SimRank~\cite{simrank} and its variants have been
proposed, but since they all try to solve the all-to-all similarity
problem directly, their scalability suffers for massive graphs.
In our approach we instead approximate the problem by focusing
our computation on local neighborhoods, dramatically reducing the
computational cost to arrive at an approximate solution, and perform
the secondary step to uncover semantic concepts from the similarity graphs.

The second model we focus on in this thesis are decision trees,
both in the batch and online domain, which we motivate through our
work on a real-world application for session length prediction in music streaming.
In the batch domain we investigate
gradient boosted trees, whose popularity has also led
to the development of multiple scalable systems like XGBoost~\cite{xgboost},
LightGBM~\cite{lightgbm}, and CatBoost~\cite{catboost}. What all these
systems have in common is that they only allow scale-out across
the data point dimension.  This means that in order to scale learning
for high-dimensional data we need to scale \emph{up} to bigger, more expensive
machines, a limitation which we address in our research. Apart from
batch decision trees, online decision trees have been proposed
to bring the accuracy and interpretability of decision trees to
the online domain~\cite{vfdt}. However, the concessions made to
bring the learning algorithm to the online domain have led to
decreased accuracy in practice and do not allow us to estimate the uncertainty
in the predictions. In our research we tackle the accuracy problem for
larges-scale streaming data by developing efficient online boosted trees,
and propose online algorithms to estimate the uncertainty in the
predictions of tree ensembles.

These challenges set the stage for this thesis: We propose scalable algorithms
for learning from massive data efficiently. We utilize different techniques
to deal with the data size including: 1) designing efficient parallel and distributed
algorithms, 2) making use of approximations to reduce the computational
cost, 3) using approximate data structures to bound the memory and computational cost
of continuously updating the models. Throughout the development of this thesis the guiding
principle has been to design algorithms that can scale regardless of data size.
This is realized through algorithms that exhibit linear scale-out characteristics,
bounding the memory and computational costs through approximations, and designing
algorithms that are optimized for distributed settings through communication-efficient
learning.


\subsubsection*{Open Challenges in Scalable Machine Learning}

To summarize, the open challenges
we aim to tackle in this dissertation are:

\begin{itemize}
	\item \emph{Scalable vertex similarity and concept discovery}. All-to-all vertex similarity
	methods like SimRank~\cite{simrank} do not scale to massive graphs. Domain-specific alternatives,
	such as word2vec~\cite{word2vec} for text do not provide a principled way to uncover
	concepts, relying on parametric clustering methods like k-means clustering to group related
	concepts. The similarity calculation for such embedding methods involves the
	calculation of a dot product for all pairs, making the process
	computationally costly.

	\item \emph{Uncertainty estimation for online tree models}. As ML models are being
	deployed in areas where mistakes can be costly, like autonomous driving and finance,
	the ability to quantify the uncertainty in predictions becomes key. The constantly
	changing environment and real-time demands of such applications indicate that
	online learning methods can be of great use. However, existing online learning
	methods that can provide uncertainty estimates, like online quantile linear regression
	~\cite{koenker2005qr}, do not provide the necessary accuracy and cannot deal
	with highly non-linear problems. More flexible models like online decision trees
	do not currently provide a way to estimate the uncertainty in their predictions.

	\item \emph{Scalable tree boosting in high dimensions.} Boosted
	decision trees are one of the most successful models in both industry and academia
	\cite{xgboost, ctr-facebook, mcrank}.
	Their success can be attributed to their scalability, ease of use, and accuracy.
	However, the current state of the art in boosted trees has two open challenges
	in terms of their scalability.
	First, for online decision trees, all existing boosting algorithms are sequential,
	meaning that they cannot take advantage of modern parallel systems, severely limiting
	their applicability to large-scale streaming datasets. For batch decision trees on the other
	hand, current approaches can only parallelize their training along the data point
	dimension. For high-dimensional data this means that scale-out is not possible,
	requiring bigger and more expensive hardware to speed up training per feature.
	In addition, current distributed approaches use dense communication that can
	be highly inefficient for sparse, high-dimensional data~\cite{xgboost, lightgbm}.
\end{itemize}

Given these limitations of the state of the art, we formulate our research question and goals for
this thesis.

\section{Research Question \& Objectives}
\label{sec:intro-question-objectives}

The objective of this dissertation is to create scalable algorithms for machine
learning, specifically for learning with graphs and decision trees. Our approach has been to use two aspects to ensure scalability:
approximation from the algorithmic side, and parallelism from the system
side. By approximation we mean either reducing an intractable problem
to a more easily solved one that provides a useful answer, or by producing
online models which can be trained on datasets of arbitrary size using bounded
computation and memory. From the system side we develop algorithms that are parallel
or are specifically optimized for the distributed setting.
Our approach is to design solutions and provide implementations
in popular open-source frameworks like Apache Spark~\cite{spark},
MOA~\cite{samoa}, and XGBoost~\cite{xgboost}, ensuring the reproducibility
of our work, and further contributing to the open-source community (see Section \ref{sec:intro-software-contributions}).

In order to create algorithms that scale to large-scale datasets and are able to
do so efficiently in a distributed setting, we can identify three objectives that can
lead to efficient solutions \cite{vasia-thesis}:

\begin{enumerate}
	\item \textbf{O1}: Reduce the amount of computation.
	\item \textbf{O2}: Reduce the amount of communication.
	\item \textbf{O3}: Bound the space cost.
\end{enumerate}

Chapters \ref{ch:concepts}, \ref{ch:uncertain-trees}, and \ref{ch:boosted-trees}
are examples where approximation has been used to make a computationally intractable
approach scalable, reducing the total amount of computation (\textbf{O1}).
In Chapter \ref{ch:concepts} for example, we provide an approximate solution
to the all-to-all graph node similarity problem by using the structure of the
graph to limit the computational cost. While the produced result cannot give
us a similarity score between any two arbitrary nodes in a graph, the result captures
much of the relevant information and provides results that can be of use
to practitioners.

Chapters \ref{ch:concepts} and \ref{ch:boosted-trees} describe learning algorithms that
are designed from the ground up with the distributed setting in mind. As network
communication is a sparse resource in shared cluster environments \cite{optimization-communication-complexity},
our algorithms minimize communication cost while providing valid
solutions for the problems of graph node similarity calculation and
boosted tree learning (\textbf{O2}).

Chapter \ref{ch:uncertain-trees} and Section \ref{sec:boostvht-method}
are examples of online learning models
that were explicitly designed to use bounded compute and memory regardless of the
dataset size (\textbf{O3}). For example, Chapter \ref{ch:uncertain-trees} provides online learning approaches
for methods that previously only existed in a batch setting, where the data were assumed to be
static and bounded, making it possible to train the algorithms on infinite streams of
data. Approximations are made in this setting as well,
as we replace exact data structures with approximate sketches, and trade-off consistency
guarantees for the ability to train the models online.


\subsection{Research Question}

This thesis taken as a whole constitutes an effort of finding appropriate
uses of approximation in order to make exact methods scale to datasets
with up to billions of datapoints and millions of features.
In addition, in all our work we take full advantage of current computing
capabilities,
whether that is providing parallel implementations of our algorithms,
or by designing algorithms from the
ground up to be efficient in a distributed setting.

These two aspects form the connecting thread between the papers
included in the thesis: Use of approximation to make previously expensive approaches tractable, and
proper use of parallelism to ensure we make full use of current
computing capabilities and allow for scale-out with the problem size.

The research question that summarizes this dissertation is:

\begin{displayquote}
	\researchQuestion
\end{displayquote}

\noindent
Based on this research question, we formulate our research statement as:

\begin{displayquote}
	\emph{Approximations allow us to make otherwise computationally costly approaches scalable.
	By carefully designing our trade-offs, we can extract useful results using a fraction
	of the resources exact approaches require. These approximations combined with utilizing parallel and distributed computation lead to efficient, scalable solutions.}
\end{displayquote}

\section{Methodology}

The research strategy for this dissertation is to perform quantitative, empirical research.
Our approach has been to perform a literature review to identify the challenges in the
state of the art, identify areas where approximation and parallelism can be applied to
optimize performance, and proceed with the design and implementation of the algorithms.

The algorithm design is guided by the research objectives. Specifically, once we identify a problem
to work on, e.g. uncertainty estimation in online decision trees, we use the three objectives
(O1, O2, O3) as a guide for our optimizations. When exact solutions exist for a particular problem
or for a different domain, e.g. a solution exists for batch learning but not for the online
domain, we look to identify the approximations that will make the approach scale to massive
data efficiently. In particular we identify different kinds of optimizations:

\begin{enumerate}
	\item \emph{Replacing exact data structures with approximate ones.} This approach can be taken
	to improve both the computational and memory cost of existing approaches. Depending on the
	level of accuracy necessary for a particular estimator, approximate data structures
	can often deliver similar accuracy to exact ones, at a fraction of the computational cost.
	One example in our research described in Chapter \ref{ch:uncertain-trees} is replacing a dense array of dependent values for the
	estimation of quantiles at tree leaves with efficient quantile sketches\cite{karnin2016kll}, thereby bounding the
	memory cost of the algorithm.
	\item \emph{Using a proxy problem to tackle an intractable one.} Instead of trying to
	tackle an intractable problem directly, we can instead try to solve a
	tractable proxy problem that can inform us about the true solution, as we have done
	in Chapter \ref{ch:concepts}. There we provide an approximate efficient algorithm for the node similarity problem by localizing the computation to two-hop neighborhoods, dramatically reducing the
	amount of computation necessary to produce an approximate but informative answer.
	\item \emph{Adapting the algorithm to the data to improve accuracy and efficiency.} In many problems
	we have prior information about the data properties. For example many real-world graphs exhibit
	power-law degree distribution \cite{small-world, barabasi-small-world}, many user interaction metrics are power-law
	distributed \cite{phonecalls, faloutsos1999internet, click-stream-power-law} or exhibit extreme sparsity \cite{esl}. These properties can
	be exploited to reduce the computational and memory cost of learning algorithms as well as
	improve their accuracy. In Chapter \ref{ch:concepts} we rely on the power-law degree
	distribution of many real-world graphs to provide approximations with measurable error,
	making an otherwise intractable computation efficient. In Chapter \ref{ch:session-length}
	we adapt the objective function to better model the power-law distribution of the dependent.
	In Chapter \ref{ch:boosted-trees} we make use of data sparsity to dramatically reduce
	the communication cost of distributed algorithms for gradient boosted trees.
\end{enumerate}

\subsection{Delimitations}


The extent to which we try to answer the research question is of course limited by
a number of factors. First, the two models we focus on are graphs and decision
trees. We find graphs to be powerful data structures that can encode complex
relationships and are therefore worthy of further study in terms of scalability.
Decision trees have been shown to perform well for a wide variety of tasks~\cite{hundreds-classifiers}.
They have distinct advantages like interpretability, although that is limited
for large ensembles, and have the ability to deal with missing data in a principled
manner, both important aspects in an industrial setting.
Comparing decision trees with deep learning models, the space
of architecture search in a deep network, i.e. designing a network that best fits
a particular problem, is much larger compared to tuning hyperparameters for
decision trees

Second, although we develop online learning methods that can deal
with concept drift, i.e. the relationship between the features and dependent
can change over time, we do not propose any new concept drift adaptation
mechanism. Instead we rely on existing concept drift adaptation mechanisms
provided by the underlying online learning algorithms in our ensembles.

Third, for our decision tree algorithms, our comparisons focus on
competing decision tree approaches. For example, for the online uncertainty
estimation problem in Chapter \ref{ch:uncertain-trees}, online linear
models can also be used to determine prediction intervals. In our evaluation
we focus on the state of the art decision tree algorithms as
most real-world problems are highly non-linear.

Finally, as we mentioned previously, our evaluation is experimental
and focuses on empirical results drawn from quantitative and qualitative
experiments. For example in Chapter \ref{ch:uncertain-trees} we trade-off
theoretical consistency guarantees for efficient computation
and only demonstrate empirically that the algorithms are able to maintain validity.

\subsection{Methodological Challenges}

During the development of this thesis we have encountered several methodological
challenges where a choice had to be made that delineates the extent to which
our research tackles the original research question. We list these challenges
here in order to make clear the limitations of this dissertation.

\begin{enumerate}
	\item \emph{Evaluation of unsupervised models.} In Chapter \ref{ch:concepts} we present our
	work on graph vertex similarity and uncovering concepts. The main methodological
	challenge for this work is the evaluation of the model's quality, as the vertex
	similarity task is unsupervised, because establishing the ground truth similarity would
	require extensive user studies, and it is impossible for the scale of the problems
	we are trying to tackle~\cite{simrank}. Instead, to provide a quantitative evaluation
	of the algorithm we need to rely on proxy tasks, such as determining the similarity
	of words using the Wordsim-353 dataset \cite{wordsim} and our model's agreement
	with other established similarity models like GloVe~\cite{glove}.

	\item \emph{Fair comparison of learning systems.} As our research has focused on optimizations,
	a major factor in evaluating the performance of an approach is the execution time. One challenge
	in assuring the fair comparison of our methods against existing algorithms and systems is to isolate
	the source of any potential gains in performance. For instance, in Chapter
	\ref{ch:uncertain-trees} we present a comparison of our approach with the current
	state-of-the-art algorithm, Mondrian Forests. Due to the complexity of Mondrian Forests
	we did not re-implement them from scratch in the same framework we used to develop our
	algorithm, introducing possible uncertainties in the runtime comparison.
	However, we make sure to demonstrate through the theoretical computational cost of the
	algorithm that our method scales much better as the number of instances grow.
	In subsequent work presented in Section \ref{sec:block-gbt} we implemented every
	algorithm from scratch to eliminate such discrepancies.

	\item \emph{Reliance on underlying frameworks.} To ease development and ensure
	our work can have a wider impact we have used established ML frameworks to develop
	our work. While using such frameworks can speed up development, it also introduces possible inefficiencies
	in our learning pipeline. For example, our use of Apache Spark \cite{spark}
	in Chapter \ref{ch:concepts} required configuration and tuning of system
	parameters to achieve high performance. Similarly, the use of MOA for
	Chapter \ref{ch:uncertain-trees} and SAMOA for Section \ref{sec:boostvht-method}
	introduces possible inefficiencies as these frameworks are meant to provide
	generic algorithm development platforms, and therefore lack optimizations for specific algorithms. Implementing our algorithms from the
	ground up in a low-level language would potentially bring further runtime
	optimizations, but would introduce many sources of error and slow down development, especially
	for error-prone distributed algorithms.
\end{enumerate}

\section{Contributions}

The main contributions of this dissertation, along with the papers they appear in,
are listed below:

\begin{itemize}
	\item \textbf{\conceptsicdm}: \emph{Knowing an object by the company it keeps: A domain-agnostic scheme for similarity discovery.}
	Olof G\"{o}rnerup, Daniel Gillblad, and Theodore Vasiloudis. 2015.
	\emph{Proceedings of the 2015 IEEE International Conference on Data Mining (ICDM '15)}, pages 121-130, 2015. \\
	\textbf{\conceptskais}: \emph{Domain-Agnostic Discovery of Similarities and Concepts at Scale.}
	Olof G\"{o}rnerup, Daniel Gillblad, and Theodore Vasiloudis. In \emph{Knowledge and
	Information Systems 51(2)}, pages 531--560, 2017.

	In these works we deal with the following research questions:\\
	\emph{How to create a scalable way to calculate the similarity between nodes
	in a graph? How can we uncover semantic concepts from the resulting graphs?
	}

	All-to-all graph node similarity approaches, like the established SimRank algorithm~\cite{simrank},
	cannot scale to massive graphs due to the computational cost scaling with at least a quadratic
	factor in the number of nodes. Our approach is to approximate the all-to-all problem
	by limiting the similarity calculation to two-hop neighbors, following the
	notion of context similarity as stated by Firth in \cite{firth} as ``You shall know
	a word by the company it keeps''. By limiting the amount of computation we introduce
	approximations with controllable error allowing for similarity calculations on massive
	data. Our implementation is optimized for the distributed setting, with its most
	expensive operation being a \emph{self-join} operation that has linear speed-up and
	constant scale-up, with limited communication.

	\textbf{Contribution:} The author of this dissertation actively contributed
	in the design of this work, designed and implemented the scalable algorithm, and contributed in
	the paper writing and experiments.

	\item \textbf{\sessionlength}:
	\emph{Predicting session length in media streaming.} Theodore Vasiloudis, Hossein Vahabi, Ross Kravitz, and Valery Rashkov. In \emph{Proceedings of the 40th
	International ACM SIGIR Conference on Research and Development in Information
	Retrieval (SIGIR '17)}, pages 977--980, 2017.

	In this work we deal with the following research questions:\\
	\emph{What are the characteristics of session length distributions in media streaming?\\
	Can we use that information to effectively predict the amount of time a user will spend using
	a music streaming application at the moment they start it?}

	While session length distribution has been investigated for search
	queries and post-ad click behavior, the behavior of users in a media
	streaming service is likely to differ greatly.
	In this work we first provide an analysis of the session length distribution
	of a major online music streaming service using tools from survival analysis.
	We then develop an appropriate
	model to predict session length from a number of features including
	user-based and contextual, session-based features.
	We demonstrate the differences in the way that user sessions evolve and end,
	and illustrate the importance of selecting an appropriate
	objective function for a non-negative, power-law distributed dependent value.
	This work acts as a use-case for our follow-up work, as it motivates the use of
	uncertainty estimation, online learning, and large-scale distributed learning with gradient boosted
	trees, topics we subsequently worked on in Papers \uncertaintreesNum, \boostvhtNum,
	and \blockgbtNum respectively.

	\textbf{Contribution:} The author of this dissertation designed and implemented the work
	presented the paper, performed the experiments, and contributed most of the
	text plus all visualizations.

	\item \textbf{\uncertaintrees}: \emph{Quantifying Uncertainty in Online Regression Forests}. Theodore Vasiloudis, Gianmarco De Francisci Morales, and Henrik Bostr\"{o}m. \emph{Under Review}.

	In this work we deal with the following research question:
	\emph{How can we estimate the uncertainty in the predictions of online tree
	ensemble methods?}

	Uncertainty estimation is of paramount importance when applying learning methods
	to domains where mistakes can be costly, like finance and autonomous vehicles.
	In addition, in these domains we have a constantly changing environment,
	and learning algorithms need to be able to constantly adapt. While ensembles
	of decision trees have been shown to be accurate estimators~\cite{hundreds-classifiers}, their online
	counterparts lack any way to estimate the uncertainty in their predictions.
	In this paper we develop two general algorithms for uncertainty estimation
	from online random forest ensembles, adapting batch methods to the online
	domain through approximate computation.

	\textbf{Contribution:} The author of this dissertation designed and implemented the
	work presented the paper, performed the experiments, and contributed most of the text plus all visualizations.

	\item \textbf{\boostvht}: \emph{BoostVHT: Boosting Distributed Streaming Decision Trees.} Theodore Vasiloudis, Foteini Beligianni, and Gianmarco De Francisci Morales. 2017.  In \emph{Proceedings of the 2017 ACM on Conference on Information and Knowledge Management (CIKM '17)}, pages 899--908, 2017.

	In this work we deal with the following research question:
	\emph{How can we make use of modern hardware to parallelize and distribute
	the computation of otherwise serial online boosting algorithms?}

	While online decision trees allow us to maintain an up-to-date scalable
	model using bounded memory and computation, the approximations they make
	can lead to decreased accuracy.
	Boosting is one of the most successful ensemble techniques to increase the
	accuracy of weak learners. However, existing online boosting approaches
	are strictly sequential, making parallelization challenging, while existing batch parallel boosting algorithms
	require the data points to be processed simultaneously, breaking the assumptions
	of the online algorithms. In this paper we bridge
	this disconnect between online and parallel boosting, by introducing
	online distributed boosting. We parallelize learning
	across the features while keeping the algorithm sequential, thereby maintaining the guarantees of online boosting
	algorithms, while at the same time providing significant speedups.

	\textbf{Contribution:} The author of this dissertation designed and implemented the
	work presented the paper, contributed to the experimentation, and contributed the majority of the text.


	\item \textbf{\blockgbt}: \emph{Block-distributed Gradient Boosted Trees}. Theodore Vasiloudis, Hyunsu Cho, and Henrik Bostr\"{o}m. To appear in \emph{Proceedings of the 42nd
	International ACM SIGIR Conference on Research and Development in Information
	Retrieval (SIGIR '19)}, 2019.


	In this work we deal with the following research questions:
	\emph{How can we improve the scalability of distributed gradient boosted tree learning
	in high dimensions?\\
	Can we add another scale-out dimension while keeping the
	communication costs manageable?}

	Gradient boosted decision trees (GBT) are one of the most popular algorithms in
	use in industry and research alike. Their popularity stems from their ability
	to deal with massive datasets, and several systems for distributed learning
	of GBTs have been developed. However one common aspect of these systems is that
	they only enable scale-out across the data point dimension, meaning that in
	order to speed-up learning per-feature we need to scale up using bigger and more
	expensive machines. In this work we demonstrate how to enable scale-out across
	both the data points \emph{and} feature dimensions, thereby allowing for scale-out
	across both, and by taking advantage of the structure of sparse datasets,
	enable faster and more cost-efficient training.

	\textbf{Contribution:} The author of this dissertation designed and implemented the
	work presented the paper, performed all the experiments, and contributed most of the text and all visualizations.
\end{itemize}

An illustration of how each of the papers listed above covers the
objectives listed in Section \ref{sec:intro-question-objectives} is given
in Table \ref{tab:papers-objectives}. We note that \sessionlength is an exploratory study that acts as motivation to our follow-up work.

\begin{table}
	\centering
	\begin{tabular}{l c c c}
		\toprule
		Publication & O1 (Computation) & O2 (Communication) & O3 (Memory) \\
		\midrule
		\conceptsicdm & \checkmark  & \checkmark  & - \\

		\conceptskais &  \checkmark  & \checkmark  & - \\

		\sessionlength & - & - & - \\

		\boostvht & - & \checkmark & \checkmark \\

		\uncertaintrees & \checkmark & - & \checkmark \\

		\blockgbt & - & \checkmark & \checkmark \\
		\bottomrule
	\end{tabular}
	\caption{Contributions by publication.}
	\label{tab:papers-objectives}
\end{table}

\subsection{Software Contributions}
\label{sec:intro-software-contributions}

As part of this dissertation we have contributed to the following open-source software:

\begin{itemize}
	\item XGBoost~\cite{xgboost} is the most popular gradient boosted trees library,
	used across academia and industry. As part of our work we have used XGBoost as the
	learning tool for \sessionlength, and have worked directly to extend XGBoost to
	support block-distribution
	for \blockgbt. As both of these works were performed during internships in private
	companies (Pandora Media and Amazon AWS respectively), we have not been able to release
	the code as open-source at the time of this writing.
	However, as a result of this work we have become actively involved in the wider
	development efforts of XGBoost and plan to integrate our block-distributed
	work in the future.
	\item MOA \cite{moa-book} is one of the most popular online learning library and includes
	support for training, evaluation and data generation for online/streaming learning.
	We have used MOA as a baseline comparison for \boostvht and extended it with
	support for uncertainty estimation among other improvements in \uncertaintrees,
	which we plan to contribute back to the project.
	\item Apache SAMOA \cite{samoa} is a library and framework that enables the development
	of distributed online learning algorithms, that can then be run on top of multiple
	stream processing engines like Apache Flink \cite{flink} and Apache Storm \cite{storm}.
	Our work on \boostvht has been developed on top of SAMOA and has been contributed back
	to the project.
\end{itemize}

Other software developed during our PhD studies that is not included in this
thesis, is the distributed machine learning library for
Apache Flink, FlinkML, which is the official ML library for Apache Flink
since the 0.9 release\footnote{\url{https://flink.apache.org/news/2015/06/24/announcing-apache-flink-0.9.0-release.html}}.

\section{Thesis Organization}

The rest of this dissertation is organized as follows:
In Part \ref{part:bg} of the dissertation we provide background information and
related work relevant to the topics covered. Chapter
\ref{ch:bg-parallel-ml} introduces some basic building blocks
for parallel and distributed machine learning, Chapter \ref{ch:bg-graph-similarity} describes the problem
of graph node similarity and existing approaches, Chapter \ref{ch:bg-decision-trees} acts as a brief introduction
in decision tree learning in the context of this dissertation, and we end the first Part with Chapter \ref{ch:bg-online-learning} that presents an overview of online learning.

Part \ref{part:results} presents the results of the work developed for this
dissertation. We have tried to make each chapter self-sufficient, providing
brief explanations of the approaches along with the most significant results.
Our work on graph node similarity from Papers \conceptsicdmNum and \conceptskaisNum
is presented in Chapter \ref{ch:concepts},
the motivating example on medias streaming session length prediction from Paper \sessionlengthNum is presented in Chapter \ref{ch:session-length}. The final
two chapters of this Part focus on decision tree learning, with Paper \uncertaintreesNum
on uncertainty estimation for online tree ensembles described in Chapter \ref{ch:uncertain-trees} and Papers \boostvhtNum and \blockgbtNum
being presented in Chapter \ref{ch:boosted-trees}, describing our work on distributed online boosted trees
and block-distributed gradient boosted trees respectively.

We conclude in Part \ref{part:conclusion} with a conclusions and discussion
chapter.
